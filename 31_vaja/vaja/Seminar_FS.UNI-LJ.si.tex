\documentclass[10pt,a4paper,oneside,fleqn]{article}
\usepackage[slovene]{babel}     % za prikaz šumnikov
\usepackage[utf8]{inputenc}     % za prikaz šumnikov
\usepackage{amsfonts}           % doda zbirko matematičnih znakov
\usepackage{graphicx}           % prikaz slik
\usepackage{longtable}          % paket za tabele, ki seraztezajo čez več strani
\usepackage{caption}
\usepackage{multirow}
\usepackage{amsmath}
\usepackage{amssymb}
\usepackage{pgfplots}
\usepackage{ctable}

% Nastavitve strani
\setlength{\oddsidemargin}{-0. cm} 
\setlength{\evensidemargin}{-0. cm}
\setlength{\topmargin}{-0.54 cm}   
\setlength{\textwidth}{16. cm}     
\setlength{\textheight}{24 cm}     
\setlength{\marginparsep}{3 mm}    
\setlength{\marginparwidth}{1.5 cm}

% Nastavitve odstavka
\setlength{\parindent}{0 mm}       
\setlength{\parskip}{1.5ex plus 0.5ex minus 0.5ex}
\renewcommand{\baselinestretch}{1.25}
\setlength{\mathindent}{1 cm}      

% Literatura
\bibliographystyle{unsrt}		% literatura je sortirana po vrstnem redu prve navedbe

% Kazalo
\setcounter{tocdepth}{2}

% Povezave, aktiven dokument
\usepackage[unicode]{hyperref}		% naredi dokument aktiven (v DVI in PDF) -> mora biti vedno zadnji pred začetkom dokumenta
%%%%%%%%%%%%%%%%%%%%%%%%%%%%%%%%%%%%%%%%%%%%%%%%%%%%%%%%%%%%%%%%%%%%%%%%%%%%%%%%%%%%%%%%%%%%%%%%%%%%
% sedaj začnemo dokument
\begin{document}

% Naslovnica
\begin{titlepage}
\begin{center}
\large UNIVERZA V LJUBLJANI
\\[1cm]
\large FAKULTETA ZA MATEMATIKO IN FIZIKO
\\[6cm]
\Large\textbf{Poročilo vaje}
\\[0.5cm]
Vaja 31 - Torzijsko nihalo 
\\[4cm]
Luka Orlić
\\[6.5cm]
Ljubljana, 25. oktober 2022
\end{center}
\end{titlepage}


%Kazalo
\tableofcontents
\pagebreak

%Tabela oznak
\section*{Seznam uporabljenih simbolov}
\addcontentsline{toc}{section}{Seznam uporabljenih simbolov} 
%
\begin{longtable}[l]{lp{.75\textwidth}} %Tabela čez več strani
\hline
%Glavna glava
Oznaka & Pomen \\
\hline
\endfirsthead
\hline
%Glava na vsaki novi strani
Oznaka & Pomen \\
\hline
\endhead
\hline
\endfoot
\hline
\endlastfoot
%Podatki po abecedi
$A_i$ & Amplituda lastnega nihaja z indeksom nihaja, enota: $deg$\\
$B_i$ & Amplituda vsiljenega nihaja z indeksom nihaja, enota: $deg$\\
$f$ & Frekvenca nihanja, enota: Hz\\
$\Phi$ & Odmik od ravnovesne lege, enota: $deg$\\
$D$ & koeficient vzmeti, enota:\\
$J$ & Vztrajnostni moment nihala, enota: $kg*m^2$\\
$\delta$ & fazni premik, enota: rad\\
$n$ & število nihajev meritve; enota: brez enote\\
$\omega$ & kotna hitrost; enota: $s^{-1}$\\

\specialrule{.1em}{.05em}{.05em} 
\end{longtable}  

\pagebreak

%Uvod
\section{Teoretični uvod}\label{sec:uvod}
\subsection{Lastno nihanje torzijskega nihala}
Nihanje je periodično gibanje okoli stabilne ravnovesne lege. nedušeno nihanje opišemo z enačbo:

\centering \Large
\begin{equation}
    \Phi = A\sin(\omega_0t); \omega_0 = \sqrt{D/J} 
\end{equation}
\raggedright \normalsize

V resnici ne obstaja nedušeno nihanje, saj se zaradi različnih uporov energija nihala manjša, to dušeno nihanje opišemo z enačbo:


\centering \Large
\begin{equation}
    \Phi = A_0e^{-\beta t}\sin(\omega_d t); \omega_d = \sqrt{\omega_0^2-\beta^2}
\end{equation}
\raggedright \normalsize

 iz tega sledi:
 
\centering \Large
\begin{equation}
    A_n = A_0e^{-\frac{2\pi\beta n}{\omega_d}}
\end{equation}
\raggedright \normalsize

z logaritmi dobimo:

\centering \Large
\begin{equation}
    \beta = \frac{\omega_d}{2\pi n}\ln{\frac{A_0}{A_n}}
\end{equation}
\raggedright \normalsize

\subsection{Vsiljeno nihanje torzijskega nihala}

Pri lastnem nihanju torzijskega nihala, po začetnim sunkom navora, frekvenco določajo lastnosti nihala ($J, D, \beta$), amplituda pa je odvisna od začetnega nihanja.
Na nihalo pa lahko deluje tudi sinusno nihajoč zunanji izvor navora, za katerega velja:

\centering \Large
\begin{equation}
    M = M_0*\sin(\omega t)
\end{equation}
\raggedright \normalsize

, se kmalu ustali s frekvenco tega navora, sicer ne v isti fazi. Takrat pravimo, da nihalo vsiljeno niha, kar opišemo z izrazom:

\centering \Large
\begin{equation}
    \Phi = B \sin(\omega t - \delta)
\end{equation}
\raggedright \normalsize

Amplituda $B$ je odvisna od amplitude $M_0$ navora M, razen tega pa od krožne frekvence $\beta$ in od lastnosti nihala. Za torzijsko nihalo z vsiljenim nihanjem velja tudi:

\centering \Large
\begin{equation}
    B = \frac{B_0}{\sqrt{[1-(\frac{\omega}{\omega_0})^2]^2 + a^2(\frac{\omega}{\omega_0})^2}}
\end{equation}
\begin{equation}
    B_0 = \frac{M_0}{D}
\end{equation}
\begin{equation}
    a = \frac{2\beta}{\omega_0}
\end{equation}
\begin{equation}
     \tan(\delta) = \frac{a(\frac{\omega}{\omega_0})}{1-(\frac{\omega}{\omega_0})^2}
\end{equation}
\raggedright \normalsize

Amplituda B je konstanta, ker dovajana moč P, s povprečno vrednostjo

\centering \Large
\begin{equation}
    \overline{P} = \frac{\omega}{2\pi} \int_0^{2\pi} Md\phi= \frac{1}{2}\omega M_0 B \sin(\delta)
\end{equation}
\raggedright \normalsize

ravno pokriva energijske izgube nihala.
Če spreminjamo frekvenco zunanjega navora $M$ , lahko zasledujemo odvisnost amplitude $B$ od frekvence. Pri majhnih ($\frac{\omega}{\omega_0} \ll 1$) je odmik nihala iz ravnovesne lege ves čas skoraj sorazmeren z navorom $M$ , amplituda pa je $B = B_0 = \frac{M_0}{D}$; fazna razlika med obema nihanjima je zelo majhna ($\delta \approx 0$). Ko povečujemo frekvenco navora, se amplituda povečuje in doseže pri $\omega \approx \omega_0$ največjo vrednost. Pravimo, de je tedaj nihalo v resonanci z navorom. Fazni premik je tedaj približno $90$ $deg$ nihalo zaostaja za navorom za četrt nihaja. Poprečna moč (glej enačbo), ki jo nihalo sprejema, je v okolici resonance največja. Pri nadaljnjem večanju frekvence začne amplituda padati, nihalo pa vedno bolj zaostaja za
navorom. Pri $\omega \ll \omega_0$ gre amplituda $B$ proti nič, fazni premik pa proti $180$ $deg$.
Krivuljo, ki nam kaže odvisnost razmerja $B/B_0$ od frekvence navora, imenujemo
resonančna krivulja 


\pagebreak

%Jedro
\graphicspath{ {./slike/} }
\section{Naloga}\label{sec:jedro}
%
\begin{enumerate}
    \item[i.)] Izmeri in izračunaj resonančno krivuljo za torzijsko nihalo pri dveh različnih dušenjih!
\end{enumerate}

\section{Potrebščine}
\begin{itemize}
    \item Torzijsko nihalo
    \item elektromotor z vzvodom
    \item Štoparica
\end{itemize}

\section{Skica}
Izjemno groba skica ekspirimentalne meritve nihanja. Skica je izjemno slaba ampak ponazarja krog, ki ga odmika polžasta vzmet.

\makebox[0pt][l]{\begin{minipage}{\textwidth}\centering\includegraphics[width=0.8\textwidth]{slike/skica.png}\captionof{figure}{Skica kalorimetra}\end{minipage}}

\section{Meritve}

\begin{itemize}
\centering
\item[]
    \begin{tabular}{|p{1.5cm}|p{1.5cm}|}
        \hline
        \multicolumn{2}{|c|}{Spološnmi podatki}\\
        \hline
        Meritev & Vrednost\\
        \hline
        $T_{sobe}$ & 22,5 C\\
        $RH$ & 57 $\%$ \\
        $P$ & 1016 hPa\\
        \hline
    \end{tabular}
\item[]
    \begin{tabular}{|p{1.5cm}|p{1.5cm}|p{1.5cm}|}
        \hline
        \multicolumn{3}{|c|}{1. Dušeno lastno nihanje meritev 1; n = 5}\\
        \hline
        Meritev & Vrednost L. & Vrednost D.\\
        \hline
        $A_z$ & 97,5 deg & 105 deg\\
        $A_k$ & 75 deg & 82,5 deg\\
        $t$ & 12,7 $s$ & 12,7 $s$\\
        \hline
    \end{tabular}
\item[]
    \begin{tabular}{|p{1.5cm}|p{1.5cm}|p{1.5cm}|}
        \hline
        \multicolumn{3}{|c|}{1. Dušeno lastno nihanje meritev 2; n = 5}\\
        \hline
        Meritev & Vrednost L. & Vrednost D\\
        \hline
        $A_z$ & 30 deg & 37,5 deg\\
        $A_k$ & 15 deg & 22,5deg\\
        $t$ & 12,8 $s$ & 12,8 $s$\\
        \hline
    \end{tabular}
\item[]
    \begin{tabular}{|p{1.5cm}|p{1.5cm}|p{1.5cm}|}
        \hline
        \multicolumn{3}{|c|}{2. Dušeno lastno nihanje meritev 1; n = 4}\\
        \hline
        Meritev & Vrednost L. & Vrednost D.\\
        \hline
        $A_z$ & 82,5 deg & 67,5 deg\\
        $A_k$ & 7.5 deg & 3,8 deg\\
        $t$ & 9,8 $s$ & 10,0 $s$\\
        \hline
    \end{tabular}
\item[]
    \begin{tabular}{|p{1.5cm}|p{1.5cm}|p{1.5cm}|}
        \hline
        \multicolumn{3}{|c|}{2. Dušeno lastno nihanje meritev 2; n = 4}\\
        \hline
        Meritev & Vrednost L. & Vrednost D\\
        \hline
        $A_z$ & 105 deg & 90 deg\\
        $A_k$ & 15 deg & 7.5 deg\\
        $t$ & 10,1 $s$ & 10,2 $s$\\
        \hline
    \end{tabular}
\item[]
    \begin{tabular}{|p{1.5cm}|p{1.5cm}|p{1.5cm}|p{1.5cm}|}
        \hline
        \multicolumn{4}{|c|}{1. Vsiljeno nihanje}\\
        \hline
        Index & Frekvenca [Hz] & Amplituda [deg] & faza\\
        \hline
        1 & 0,10 & 7,5 & 0\\
        2 & 0,20 & 9,8 & 0\\
        3 & 0,30 & 11,3 & $\frac{\pi}{16}$\\
        4 & 0,32 & 12 & $\frac{\pi}{8}$\\
        5 & 0,34 & 13,5 & $\frac{\pi}{4}$\\
        6 & 0,36 & 21 & $\frac{\pi}{2}$\\
        7 & 0,37 & 41,3 & $\frac{\pi}{2}$\\
        \hline
        8 & 0,375 & 180 & $\frac{\pi}{2}$\\
        \hline
        9 & 0,38 & 138,8 & $\frac{\pi}{2}$\\
        10 & 0,40 & 32,3 & $\frac{3\pi}{4}$\\
        11 & 0,42 & 21 & $\pi$\\
        12 & 0,44 & 15 & $\pi$\\
        13 & 0,50 & 7,5 & $\pi$\\
        14 & 0,60 & 3,8 & $\pi$\\
        \hline
    \end{tabular}
\item[]
    \begin{tabular}{|p{1.5cm}|p{1.5cm}|p{1.5cm}|p{1.5cm}|}
        \hline
        \multicolumn{4}{|c|}{2. Vsiljeno nihanje}\\
        \hline
        Index & Frekvenca [Hz] & Amplituda [deg] & faza\\
        \hline
        1 & 0,10 & 6 & 0\\
        2 & 0,20 & 7,5 & 0\\
        3 & 0,30 & 9,8 & 0\\
        4 & 0,32 & 11,3 & $\frac{\pi}{4}$\\
        5 & 0,34 & 13,5 & $\frac{\pi}{4}$\\
        6 & 0,36 & 17,3 & $\frac{\pi}{2}$\\
        7 & 0,38 & 22,5 & $\frac{\pi}{2}$\\
        8 & 0,39 & 30 & $\frac{\pi}{2}$\\
        \hline
        9 & 0,40 & 37,5 & $\frac{\pi}{2}$\\
        \hline
        10 & 0,41 & 22,5 & $\frac{\pi}{2}$\\
        11 & 0,42 & 21 & $\frac{\pi}{2}$\\
        12 & 0,44 & 13,5 & $\frac{3\pi}{4}$\\
        13 & 0,46 & 9,8 & $\frac{3\pi}{4}$\\
        14 & 0,50 & 7,5 & $\pi$\\
        15 & 0,60 & 3,8 & $\pi$\\
        \hline
    \end{tabular}
\item[*]
    faza je fazni zamik
\end{itemize}
\subsection{Metodologija}

Podatke za splošne pogoje smo pridobili s pomočjo stenskega aparata za merjenje rleativne vlažnosti, pritiska in teperature. Pogoji v sobi so ob začetku in kocu bili enaki. S pomočjo posebnega elektromotorja smo na vzmet inducirali sinusno stiskanje in raztezanje vzmeti, odmik smo merili vizualno. Privzeli smo, da je poskus ponovljiv, ter da da vedno iste rezultate, zato smo posebaj merili čas petih nihajev ter amplitudo petega nihaja. Tu se lahko pojavi sistemska napaka. Posebno smo merili vrednosti za desni ekstrem nihala ter levi ekstrem nihala.

\section{Obdelava meritev}
Za izračun lastne dušene frekvence nihala smo uporabili smo uporabili naslednjo enačbo, da smo dobili rezultate:

\subsection{Lastna dušena frekvenca brez dodatnega dušila}

\centering \Large
\begin{equation}
    \omega_d = \frac{2\pi}{t_0}; t_0 = \frac{\overline{t}}{n}
\end{equation}
\raggedright \normalsize
\centering \Large
\begin{equation}
    \omega_d = \frac{2\pi n}{\overline{t}}
\end{equation}
\raggedright \normalsize
\centering \Large
\begin{equation}
    \omega_d = 2,5 * (1\pm 0,004) s^{-1}
\end{equation}
\raggedright \normalsize

\subsection{Lastna dušena frekvenca z dodatnim dušilom}

Uporabimo enačbe iz prejšnjega računa ter dobimo:

\centering \Large
\begin{equation}
    \omega_d = 2,5 * (1\pm 0,02) s^{-1}
\end{equation}
\raggedright \normalsize

\subsection{Koeficient dušena brez dodatnega dušila}

\centering \Large
\begin{equation}
    \beta = \frac{\omega_d}{2\pi n}\ln{\frac{A_0}{A_n}}
\end{equation}
\raggedright \normalsize
\centering \Large
\begin{equation}
    \overline{\beta} = 0,03 * (1\pm0) s^{-1}
\end{equation}
\raggedright \normalsize

Opozorilo: napaka pri $\overline{\beta}$ je reda $10^{-5}$ in ne zares $0$.

\subsection{Koeficient dušena z dodatnim dušilom}

\centering \Large
\begin{equation}
    \beta = \frac{\omega_d}{2\pi n}\ln{\frac{A_0}{A_n}}
\end{equation}
\raggedright \normalsize
\centering \Large
\begin{equation}
    \overline{\beta} = 0,25 * (1\pm0,01) s^{-1}
\end{equation}
\raggedright \normalsize

\subsection{Lastna frekvenca nihala brez dodatnega dušila}
\centering \Large
\begin{equation}
    \omega_0 = \sqrt{\omega_d^2+\beta^2}
\end{equation}
\raggedright \normalsize
\centering \Large
\begin{equation}
    \omega_0 = 2,5 * (1\pm0,01) s^{-1}
\end{equation}
\raggedright \normalsize

\subsection{Lastna frekvenca nihala z dodatnim dušilom}
\centering \Large
\begin{equation}
    \omega_0 = \sqrt{\omega_d^2+\beta^2}
\end{equation}
\raggedright \normalsize
\centering \Large
\begin{equation}
    \omega_0 = 2,5 * (1\pm 0,04) s^{-1}
\end{equation}
\raggedright \normalsize

\subsection{Resonančna krivulja}

Y-os sestavlja $\frac{B}{B_0}$ ter X-os sestavlja $\frac{\omega}{\omega_0}$ torej graf $\frac{B}{B_0}$($\frac{\omega}{\omega_0}$)

\centering \Large
\begin{equation}
    \frac{\omega}{\omega_0} = \frac{f*2\pi}{\omega_0}
\end{equation}
\raggedright \normalsize

in

\centering \Large
\begin{equation}
    \frac{B}{B_0} = \frac{1}{\sqrt{(1-{(\frac{\omega}{\omega_0})}^2)^2+a^2({\frac{\omega}{\omega_0}})^2}}
\end{equation}
\raggedright \normalsize

koordinata izmerjene točke na grafu je torej:

\centering \Large
\begin{equation}
    (\frac{f*2\pi}{\omega_0} , \frac{1}{\sqrt{(1-{(\frac{\omega}{\omega_0})}^2)^2+a^2({\frac{\omega}{\omega_0}})^2}})
\end{equation}
\raggedright \normalsize

koordinata idealne resonančne krivulje pa se računa kot:

\centering \Large
\begin{equation}
    (\frac{\omega}{\omega_0} , \frac{1}{\sqrt{(1-{(\frac{\omega}{\omega_0})}^2)^2+a^2({\frac{\omega}{\omega_0}})^2}})
\end{equation}
\raggedright \normalsize

\centering \large
\begin{tabular}{|p{1.5cm}|p{1.5cm}|p{1.5cm}|}
    \hline
    \multicolumn{3}{|c|}{1. Vsiljeno nihanje}\\
    \hline
    Index & $\frac{\omega}{\omega_0}$ & $\frac{B}{B_0}$ \\
    \hline
    1 & 0,25 & 1,07\\
    2 & 0,50 & 1,34\\
    3 & 0,75 & 2,31\\
    4 & 0,80 & 2,82\\
    5 & 0,86 & 4,40\\
    6 & 0,91 & 5,47 \\
    7 & 0,93 & 7,28\\
    8 & 0,94 & 8,75\\
    9 & 0,96 & 10,98\\
    10 & 1,01 & 38,09\\
    11 & 1,06 & 8,57\\
    12 & 1,11 & 4,46\\
    13 & 1,26 & 1,73\\
    14 & 1,51 & 0,78\\
    \hline
\end{tabular}

\begin{tabular}{|p{1.5cm}|p{1.5cm}|p{1.5cm}|}
    \hline
    \multicolumn{3}{|c|}{2. Vsiljeno nihanje}\\
    \hline
    Index & $\frac{\omega}{\omega_0}$ & $\frac{B}{B_0}$ \\
    \hline
    1 & 0,25 & 1,04 \\
    2 & 0,50 & 1,29\\
    3 & 0,75 & 2,10\\
    4 & 0,80 & 2,46\\
    5 & 0,86 & 2,98\\
    6 & 0,91 & 3,70\\
    7 & 0,96 & 4,58\\
    8 & 0,98 & 4,91\\
    9 & 1,01 & 4,99\\
    10 & 1,03 & 4,78\\
    11 & 1,06 & 4,34\\
    12 & 1,11 & 3,34\\
    13 & 1,16 & 2,55\\
    14 & 1,26 & 1,63\\
    15 & 1,51 & 0,56\\
    \hline
\end{tabular}

\large \centering
\begin{tikzpicture}
\begin{axis}[
    title={Resonančna frekvenca brez dodatnega dušenja},
    xlabel={$\omega/\omega_0$},
    ylabel={$B/B_0$},
    xmin=0, xmax=1.6,
    ymin=0, ymax=40,
    xtick={0,0.2,0.4,0.6,0.8,1,1.2,1.4,1.6},
    ytick={0,5,10,15,20,25,30,35,40},
    legend pos=north west,
    ymajorgrids=true,
    xmajorgrids=true,
    grid style=dashed,
]

\addplot[
    color=blue,
    mark=square,
    ]
    coordinates {
    (	0.25	,	1.07	)
(	0.50	,	1.34	)
(	0.75	,	2.32	)
(	0.80	,	2.83	)
(	0.88	,	4.40	)
(	0.90	,	5.47	)
(	0.93	,	7.29	)
(	0.94	,	8.77	)
(	0.96	,	11.01	)
(	1.01	,	37.92	)
(	1.06	,	8.55	)
(	1.11	,	4.46	)
(	1.26	,	1.72	)
(	1.51	,	0.78	)

    };
    \legend{eksperimentalni}

\addplot[
    color=red]
    %{1.000/((0.000576*x^2 + (1.000-x^2)^2)^(1/2))};
    %{1/sqrt((1-x*x)^2 + (2*0.03/2.5)^2 * x*x)};
    coordinates {
    (	0	,	1	)
(	0.016	,	1.000255992	)
(	0.032	,	1.001024754	)
(	0.048	,	1.002308653	)
(	0.064	,	1.004111652	)
(	0.08	,	1.006439345	)
(	0.096	,	1.009298996	)
(	0.112	,	1.012699599	)
(	0.128	,	1.016651948	)
(	0.144	,	1.021168727	)
(	0.16	,	1.026264609	)
(	0.176	,	1.03195638	)
(	0.192	,	1.038263085	)
(	0.208	,	1.045206188	)
(	0.224	,	1.052809765	)
(	0.24	,	1.061100724	)
(	0.256	,	1.070109053	)
(	0.272	,	1.079868115	)
(	0.288	,	1.090414977	)
(	0.304	,	1.101790787	)
(	0.32	,	1.114041219	)
(	0.336	,	1.127216974	)
(	0.352	,	1.14137436	)
(	0.368	,	1.156575971	)
(	0.384	,	1.17289146	)
(	0.4	,	1.190398453	)
(	0.416	,	1.209183602	)
(	0.432	,	1.229343832	)
(	0.448	,	1.25098779	)
(	0.464	,	1.27423757	)
(	0.48	,	1.299230751	)
(	0.496	,	1.326122823	)
(	0.512	,	1.355090094	)
(	0.528	,	1.386333189	)
(	0.544	,	1.420081283	)
(	0.56	,	1.456597259	)
(	0.576	,	1.496184032	)
(	0.592	,	1.539192359	)
(	0.608	,	1.586030556	)
(	0.624	,	1.637176691	)
(	0.64	,	1.693194019	)
(	0.656	,	1.754750708	)
(	0.672	,	1.822645319	)
(	0.688	,	1.897840092	)
(	0.704	,	1.981504973	)
(	0.72	,	2.07507666	)
(	0.736	,	2.180338988	)
(	0.752	,	2.299534221	)
(	0.768	,	2.435520041	)
(	0.784	,	2.59199569	)
(	0.8	,	2.773835569	)
(	0.816	,	2.987594803	)
(	0.832	,	3.242299661	)
(	0.848	,	3.550728769	)
(	0.864	,	3.931580082	)
(	0.88	,	4.413326781	)
(	0.896	,	5.041514376	)
(	0.912	,	5.893665147	)
(	0.928	,	7.112813822	)
(	0.944	,	8.99308764	)
(	0.96	,	12.23759994	)
(	0.976	,	18.90592414	)
(	0.992	,	34.90498226	)
(	1.008	,	34.43560144	)
(	1.024	,	18.36917508	)
(	1.04	,	11.71892395	)
(	1.056	,	8.482311338	)
(	1.072	,	6.605619853	)
(	1.088	,	5.388217752	)
(	1.104	,	4.53690996	)
(	1.12	,	3.909057589	)
(	1.136	,	3.427326591	)
(	1.152	,	3.046269411	)
(	1.168	,	2.737467583	)
(	1.184	,	2.482255461	)
(	1.2	,	2.267874341	)
(	1.216	,	2.085310645	)
(	1.232	,	1.928017518	)
(	1.248	,	1.791126018	)
(	1.264	,	1.6709405	)
(	1.28	,	1.564605616	)
(	1.296	,	1.469880539	)
(	1.312	,	1.384982224	)
(	1.328	,	1.308474277	)
(	1.344	,	1.239186682	)
(	1.36	,	1.176156829	)
(	1.376	,	1.118585544	)
(	1.392	,	1.06580384	)
(	1.408	,	1.017247463	)
(	1.424	,	0.972437182	)
(	1.44	,	0.930963375	)
(	1.456	,	0.892473837	)
(	1.472	,	0.856664073	)
(	1.488	,	0.823269488	)
(	1.504	,	0.792059069	)
(	1.52	,	0.762830224	)
(	1.536	,	0.735404546	)
(	1.552	,	0.709624315	)
(	1.568	,	0.685349586	)
(	1.584	,	0.662455757	)
(	1.6	,	0.640831525	)
};
    
\addlegendentry{idealni}
    
\end{axis}
\end{tikzpicture}
\begin{tikzpicture}
\begin{axis}[
    title={Resonančna frekvenca z dodatnim dušenjem},
    xlabel={$\omega/\omega_0$},
    ylabel={$B/B_0$},
    xmin=0, xmax=1.6,
    ymin=0, ymax=6,
    xtick={0,0.2,0.4,0.6,0.8,1,1.2,1.4,1.6},
    ytick={0,1,2,3,4,5,6},
    legend pos=north west,
    ymajorgrids=true,
    xmajorgrids=true,
    grid style=dashed,
]

\addplot[
    color=blue,
    mark=square,
    ]
    coordinates {
    	(	0.25	,	1.07	)
(	0.50	,	1.33	)
(	0.75	,	2.19	)
(	0.80	,	2.58	)
(	0.85	,	3.13	)
(	0.90	,	3.90	)
(	0.96	,	4.76	)
(	0.98	,	5.00	)
(	1.01	,	4.97	)
(	1.03	,	4.65	)
(	1.06	,	4.17	)
(	1.11	,	3.18	)
(	1.16	,	2.45	)
(	1.26	,	1.58	)
(	1.51	,	0.76	)


    };
    \legend{eksperimentalni}

\addplot[
    color=red]
    %{1.000/((0.000576*x^2 + (1.000-x^2)^2)^(1/2))};
    %{1/sqrt((1-x*x)^2 + (2*0.03/2.5)^2 * x*x)};
    coordinates {(	0	,	1	)
(	0.016	,	1.000250942	)
(	0.032	,	1.001004507	)
(	0.048	,	1.002262924	)
(	0.064	,	1.004029922	)
(	0.08	,	1.00631076	)
(	0.096	,	1.009112267	)
(	0.112	,	1.012442888	)
(	0.128	,	1.016312753	)
(	0.144	,	1.020733746	)
(	0.16	,	1.025719601	)
(	0.176	,	1.031286007	)
(	0.192	,	1.037450731	)
(	0.208	,	1.044233762	)
(	0.224	,	1.051657472	)
(	0.24	,	1.059746809	)
(	0.256	,	1.068529507	)
(	0.272	,	1.07803633	)
(	0.288	,	1.088301357	)
(	0.304	,	1.09936229	)
(	0.32	,	1.111260826	)
(	0.336	,	1.124043064	)
(	0.352	,	1.137759982	)
(	0.368	,	1.152467973	)
(	0.384	,	1.168229471	)
(	0.4	,	1.185113658	)
(	0.416	,	1.203197286	)
(	0.432	,	1.222565625	)
(	0.448	,	1.243313552	)
(	0.464	,	1.265546824	)
(	0.48	,	1.289383554	)
(	0.496	,	1.314955924	)
(	0.512	,	1.342412206	)
(	0.528	,	1.371919114	)
(	0.544	,	1.403664579	)
(	0.56	,	1.437861028	)
(	0.576	,	1.474749246	)
(	0.592	,	1.514602979	)
(	0.608	,	1.557734384	)
(	0.624	,	1.604500537	)
(	0.64	,	1.65531119	)
(	0.656	,	1.710638022	)
(	0.672	,	1.77102568	)
(	0.688	,	1.837104888	)
(	0.704	,	1.909607946	)
(	0.72	,	1.989386841	)
(	0.736	,	2.077434023	)
(	0.752	,	2.174905473	)
(	0.768	,	2.283144826	)
(	0.784	,	2.403705605	)
(	0.8	,	2.538365413	)
(	0.816	,	2.689119994	)
(	0.832	,	2.858134321	)
(	0.848	,	3.047609039	)
(	0.864	,	3.259489153	)
(	0.88	,	3.494893916	)
(	0.896	,	3.75308722	)
(	0.912	,	4.029774241	)
(	0.928	,	4.314625911	)
(	0.944	,	4.588468732	)
(	0.96	,	4.821836181	)
(	0.976	,	4.978139469	)
(	0.992	,	5.024141468	)
(	1.008	,	4.944644765	)
(	1.024	,	4.751000329	)
(	1.04	,	4.475603205	)
(	1.056	,	4.157231994	)
(	1.072	,	3.828546045	)
(	1.088	,	3.511223455	)
(	1.104	,	3.216896475	)
(	1.12	,	2.950181633	)
(	1.136	,	2.711553305	)
(	1.152	,	2.499365969	)
(	1.168	,	2.31107748	)
(	1.184	,	2.143917133	)
(	1.2	,	1.995217211	)
(	1.216	,	1.862556524	)
(	1.232	,	1.74380534	)
(	1.248	,	1.637122017	)
(	1.264	,	1.540928363	)
(	1.28	,	1.453877645	)
(	1.296	,	1.374822055	)
(	1.312	,	1.302782677	)
(	1.328	,	1.236923064	)
(	1.344	,	1.176526619	)
(	1.36	,	1.120977505	)
(	1.376	,	1.069744677	)
(	1.392	,	1.022368587	)
(	1.408	,	0.978450129	)
(	1.424	,	0.937641439	)
(	1.44	,	0.899638248	)
(	1.456	,	0.864173493	)
(	1.472	,	0.831011975	)
(	1.488	,	0.799945884	)
(	1.504	,	0.770791028	)
(	1.52	,	0.743383653	)
(	1.536	,	0.717577748	)
(	1.552	,	0.693242755	)
(	1.568	,	0.670261621	)
(	1.584	,	0.648529125	)
(	1.6	,	0.627950449	)
};
    
\addlegendentry{idealni}
    
\end{axis}
\end{tikzpicture}
\raggedright \normalsize


\subsection{Graf faznega zamika $\delta$($\omega/\omega_0$)}

Fazni zamik računamo s formulo:

\centering \large
\begin{equation}
     \tan(\delta) = \frac{a(\frac{\omega}{\omega_0})}{1-(\frac{\omega}{\omega_0})^2}
\end{equation}
\raggedright \normalsize

Tako bomo iz $\omega/\omega_0$ določili pravilni $\delta$, ter to primerjali z grafom funkcije, zgoraj omenjene enačbe, če je $\omega/\omega_0 = x$ 

\centering \large
\begin{tabular}{|p{1.5cm}|p{1.5cm}|p{1.5cm}|}
    \hline
    \multicolumn{3}{|c|}{1. Fazni zamik}\\
    \hline
    Index & $\delta_e$ & $\delta_i$ \\
    \hline
    1 & 0 & 0,06\\
    2 & 0 & 0,16\\
    3 & $\pi/16$ & 0,40\\
    4 & $\pi/8$ & 0,50\\
    5 & $\pi/4$ & 0,75\\
    6 & $\pi/2$ & 0,87 \\
    7 & $\pi/2$ & 1,03\\
    8 & $\pi/2$ & 1,11\\
    9 & $\pi/2$ & 1,20\\
    10 & $3\pi/4$ & 1,61\\
    11 & $\pi$ & 1,99\\
    12 & $\pi$ & 2,27\\
    13 & $\pi$ & 2,66\\
    14 & $\pi$ & 2,68\\
    \hline
\end{tabular}

\begin{tabular}{|p{1.5cm}|p{1.5cm}|p{1.5cm}|}
    \hline
    \multicolumn{3}{|c|}{2. Fazni zamik}\\
    \hline
    Index & $\delta_e$ & $\delta_i$ \\
    \hline
    1 & 0 & 0,05 \\
    2 & 0 & 0,13\\
    3 & 0 & 0,34\\
    4 & $\pi/4$ & 0,43\\
    5 & $\pi/4$ & 0,56\\
    6 & $\pi/2$ & 0,78\\
    7 & $\pi/2$ & 1,14\\
    8 & $\pi/2$ & 1,37\\
    9 & $\pi/2$ & 1,62\\
    10 & $\pi/2$ & 1,86\\
    11 & $\pi/2$ & 2,07\\
    12 & $3\pi/4$ & 2,36\\
    13 & $3\pi/4$ & 2,54\\
    14 & $\pi$ & 2,73\\
    15 & $\pi$ & 2,91\\
    \hline
\end{tabular}

\raggedright \normalsize

\begin{tikzpicture}
\begin{axis}[
    title={Fazni zamik brez dodatnega dušenja},
    xlabel={$\omega/\omega_0$},
    ylabel={$\delta$},
    xmin=0, xmax=1.6,
    ymin=0, ymax=3.5,
    xtick={0,0.2,0.4,0.6,0.8,1,1.2,1.4,1.6},
    ytick={0,0.5,1.0,1.5,2.0,2.5,3.0,3.5},
    legend pos=north west,
    ymajorgrids=true,
    xmajorgrids=true,
    grid style=dashed,
]

\addplot[
    color=blue,
    mark=square,
    ]
    coordinates { (	0.25	,	0	)
(	0.50	,	0	)
(	0.75	,	0.19634954084)
(	0.80	,	0.39269908169)
(	0.88	,	0.78539816339 	)
(	0.90	,	1.57079632679 	)
(	0.93	,	1.57079632679 	)
(	0.94	,	1.57079632679 	)
(	0.96	,	1.57079632679 	)
(	1.01	,	 2.35619449019 	)
(	1.06	,	3.14159265359 	)
(	1.11	,	3.14159265359 	)
(	1.26	,	3.14159265359 	)
(	1.51	,	3.14159265359 	)

    
    };
    \legend{eksperimentalni}

\addplot[
    color=red
    ]
    coordinates { (	0	,	0	)
(	0.016	,	0.003840964	)
(	0.032	,	0.007687721	)
(	0.048	,	0.01154609	)
(	0.064	,	0.015421951	)
(	0.08	,	0.019321267	)
(	0.096	,	0.023250121	)
(	0.112	,	0.027214745	)
(	0.128	,	0.031221551	)
(	0.144	,	0.03527717	)
(	0.16	,	0.039388485	)
(	0.176	,	0.043562674	)
(	0.192	,	0.047807256	)
(	0.208	,	0.05213013	)
(	0.224	,	0.056539635	)
(	0.24	,	0.061044604	)
(	0.256	,	0.065654424	)
(	0.272	,	0.07037911	)
(	0.288	,	0.075229383	)
(	0.304	,	0.080216752	)
(	0.32	,	0.085353618	)
(	0.336	,	0.090653382	)
(	0.352	,	0.096130573	)
(	0.368	,	0.101800986	)
(	0.384	,	0.107681849	)
(	0.4	,	0.113792007	)
(	0.416	,	0.120152137	)
(	0.432	,	0.126784994	)
(	0.448	,	0.133715694	)
(	0.464	,	0.140972049	)
(	0.48	,	0.148584946	)
(	0.496	,	0.1565888	)
(	0.512	,	0.165022084	)
(	0.528	,	0.173927944	)
(	0.544	,	0.183354936	)
(	0.56	,	0.193357897	)
(	0.576	,	0.203998977	)
(	0.592	,	0.215348884	)
(	0.608	,	0.227488365	)
(	0.624	,	0.240509998	)
(	0.64	,	0.254520358	)
(	0.656	,	0.269642639	)
(	0.672	,	0.286019851	)
(	0.688	,	0.303818711	)
(	0.704	,	0.323234402	)
(	0.72	,	0.344496384	)
(	0.736	,	0.36787547	)
(	0.752	,	0.393692426	)
(	0.768	,	0.422328293	)
(	0.784	,	0.454236593	)
(	0.8	,	0.489957326	)
(	0.816	,	0.530132206	)
(	0.832	,	0.575519582	)
(	0.848	,	0.627005705	)
(	0.864	,	0.685605767	)
(	0.88	,	0.752443066	)
(	0.896	,	0.828687358	)
(	0.912	,	0.915425789	)
(	0.928	,	1.013437802	)
(	0.944	,	1.122863174	)
(	0.96	,	1.242808847	)
(	0.976	,	1.37103745	)
(	0.992	,	1.50396054	)
(	1.008	,	1.637101106	)
(	1.024	,	1.765937268	)
(	1.04	,	1.886766599	)
(	1.056	,	1.997214515	)
(	1.072	,	2.09626843	)
(	1.088	,	2.183985462	)
(	1.104	,	2.261099196	)
(	1.12	,	2.328678501	)
(	1.136	,	2.387892019	)
(	1.152	,	2.439870983	)
(	1.168	,	2.485642322	)
(	1.184	,	2.52610482	)
(	1.2	,	2.562028668	)
(	1.216	,	2.594066078	)
(	1.232	,	2.622766041	)
(	1.248	,	2.648589621	)
(	1.264	,	2.671924108	)
(	1.28	,	2.6930954	)
(	1.296	,	2.712378513	)
(	1.312	,	2.730006326	)
(	1.328	,	2.746176802	)
(	1.344	,	2.761058907	)
(	1.36	,	2.774797464	)
(	1.376	,	2.787517117	)
(	1.392	,	2.799325588	)
(	1.408	,	2.810316351	)
(	1.424	,	2.820570839	)
(	1.44	,	2.830160261	)
(	1.456	,	2.839147118	)
(	1.472	,	2.847586461	)
(	1.488	,	2.855526943	)
(	1.504	,	2.863011706	)
(	1.52	,	2.870079124	)
(	1.536	,	2.87676343	)
(	1.552	,	2.883095254	)
(	1.568	,	2.889102075	)
(	1.584	,	2.894808614	)
(	1.6	,	2.900237163	)

    
    };
    
\addlegendentry{idealni}
    
\end{axis}
\end{tikzpicture}

\begin{tikzpicture}
\begin{axis}[
    title={Fazni zamik z dodatnim dušenjem},
    xlabel={$\omega/\omega_0$},
    ylabel={$\delta$},
    xmin=0, xmax=1.6,
    ymin=0, ymax=3.5,
    xtick={0,0.2,0.4,0.6,0.8,1,1.2,1.4,1.6},
    ytick={0,0.5,1.0,1.5,2.0,2.5,3.0,3.5},
    legend pos=north west,
    ymajorgrids=true,
    xmajorgrids=true,
    grid style=dashed,
]

\addplot[
    color=blue,
    mark=square,
    ]
    coordinates { (	0.25	,	0	)
(	0.50	,	0	)
(	0.75	,	0	)
(	0.80	,	0.78539816339	)
(	0.85	,	0.78539816339	)
(	0.90	,	1.57079632679	)
(	0.96	,	1.57079632679	)
(	0.98	,	1.57079632679	)
(	1.01	,	1.57079632679	)
(	1.03	,	1.57079632679	)
(	1.06	,	1.57079632679	)
(	1.11	,	2.35619449019	)
(	1.16	,	2.35619449019	)
(	1.26	,	3.14159265359	)
(	1.51	,	3.14159265359	)

    };
    \legend{eksperimentalni}

\addplot[
    color=red
    ]
    coordinates { (	0	,	0	)
(	0.016	,	0.003840964	)
(	0.032	,	0.007687721	)
(	0.048	,	0.01154609	)
(	0.064	,	0.015421951	)
(	0.08	,	0.019321267	)
(	0.096	,	0.023250121	)
(	0.112	,	0.027214745	)
(	0.128	,	0.031221551	)
(	0.144	,	0.03527717	)
(	0.16	,	0.039388485	)
(	0.176	,	0.043562674	)
(	0.192	,	0.047807256	)
(	0.208	,	0.05213013	)
(	0.224	,	0.056539635	)
(	0.24	,	0.061044604	)
(	0.256	,	0.065654424	)
(	0.272	,	0.07037911	)
(	0.288	,	0.075229383	)
(	0.304	,	0.080216752	)
(	0.32	,	0.085353618	)
(	0.336	,	0.090653382	)
(	0.352	,	0.096130573	)
(	0.368	,	0.101800986	)
(	0.384	,	0.107681849	)
(	0.4	,	0.113792007	)
(	0.416	,	0.120152137	)
(	0.432	,	0.126784994	)
(	0.448	,	0.133715694	)
(	0.464	,	0.140972049	)
(	0.48	,	0.148584946	)
(	0.496	,	0.1565888	)
(	0.512	,	0.165022084	)
(	0.528	,	0.173927944	)
(	0.544	,	0.183354936	)
(	0.56	,	0.193357897	)
(	0.576	,	0.203998977	)
(	0.592	,	0.215348884	)
(	0.608	,	0.227488365	)
(	0.624	,	0.240509998	)
(	0.64	,	0.254520358	)
(	0.656	,	0.269642639	)
(	0.672	,	0.286019851	)
(	0.688	,	0.303818711	)
(	0.704	,	0.323234402	)
(	0.72	,	0.344496384	)
(	0.736	,	0.36787547	)
(	0.752	,	0.393692426	)
(	0.768	,	0.422328293	)
(	0.784	,	0.454236593	)
(	0.8	,	0.489957326	)
(	0.816	,	0.530132206	)
(	0.832	,	0.575519582	)
(	0.848	,	0.627005705	)
(	0.864	,	0.685605767	)
(	0.88	,	0.752443066	)
(	0.896	,	0.828687358	)
(	0.912	,	0.915425789	)
(	0.928	,	1.013437802	)
(	0.944	,	1.122863174	)
(	0.96	,	1.242808847	)
(	0.976	,	1.37103745	)
(	0.992	,	1.50396054	)
(	1.008	,	1.637101106	)
(	1.024	,	1.765937268	)
(	1.04	,	1.886766599	)
(	1.056	,	1.997214515	)
(	1.072	,	2.09626843	)
(	1.088	,	2.183985462	)
(	1.104	,	2.261099196	)
(	1.12	,	2.328678501	)
(	1.136	,	2.387892019	)
(	1.152	,	2.439870983	)
(	1.168	,	2.485642322	)
(	1.184	,	2.52610482	)
(	1.2	,	2.562028668	)
(	1.216	,	2.594066078	)
(	1.232	,	2.622766041	)
(	1.248	,	2.648589621	)
(	1.264	,	2.671924108	)
(	1.28	,	2.6930954	)
(	1.296	,	2.712378513	)
(	1.312	,	2.730006326	)
(	1.328	,	2.746176802	)
(	1.344	,	2.761058907	)
(	1.36	,	2.774797464	)
(	1.376	,	2.787517117	)
(	1.392	,	2.799325588	)
(	1.408	,	2.810316351	)
(	1.424	,	2.820570839	)
(	1.44	,	2.830160261	)
(	1.456	,	2.839147118	)
(	1.472	,	2.847586461	)
(	1.488	,	2.855526943	)
(	1.504	,	2.863011706	)
(	1.52	,	2.870079124	)
(	1.536	,	2.87676343	)
(	1.552	,	2.883095254	)
(	1.568	,	2.889102075	)
(	1.584	,	2.894808614	)
(	1.6	,	2.900237163	)

    
    };
    
\addlegendentry{idealni}
    
\end{axis}
\end{tikzpicture}

\subsection{Povprečna sprejeta moč - $\frac{\overline{P}}{M_0 B_0 \omega_0}$($\omega/\omega_0$)}

Povrpečno sprejeto moč izračunamo s formulo:

\centering \Large
\begin{equation}
    \overline{P} = \frac{1}{2}\omega M_0 B \sin{(\delta)}
\end{equation}
\raggedright \normalsize

iz česar sledi

\centering \Large
\begin{equation}
    \frac{\overline{P}}{M_0B_0\omega_O} = \frac{1}{2} \frac{\omega}{\omega_0} \frac{B}{B_0} \sin(\delta)
\end{equation}
\raggedright \normalsize

Za podatke lahko izračunamo:

\centering \large
\begin{tabular}{|p{1.5cm}|p{1.5cm}|p{1.5cm}|p{1.5cm}|p{1.5cm}|}
    \hline
    \multicolumn{5}{|c|}{1. Povprečna sprejeta moč}\\
    \hline
    Index & $\frac{\omega}{\omega_0}$ & $\frac{B}{B_0}$ & $\delta$ izr. & $\frac{\overline{P}}{M_0 B_0 \omega_0}$\\ 
    \hline
    1 & 0,25 & 1,07 & 0,06 & 0,01\\
    2 & 0,50 & 1,34 & 0,16 & 0,05\\
    3 & 0,75 & 2,32 & 0,40 & 0,34\\
    4 & 0,80 & 2,83 & 0,50 & 0,55\\
    5 & 0,86 & 4,40 & 0,75 & 1,32\\
    6 & 0,91 & 5,47 & 0,87 & 1,90\\
    7 & 0,93 & 7,29 & 1,03 & 2,90\\
    8 & 0,94 & 8,77 & 1,11 & 3,71\\
    9 & 0,96 & 11,01 & 1,20 & 4,91\\
    10 & 1,01 & 37,92 & 1,61 & 19,04\\
    11 & 1,06 & 8,55 & 1,99 & 4,11\\
    12 & 1,11 & 4,46 & 2,27 & 1,89\\
    13 & 1,26 & 1,72 & 2,66 & 0,50\\
    14 & 1,51 & 0,78 & 2,68 & 0,16\\
    \hline
\end{tabular}

\begin{tabular}{|p{1.5cm}|p{1.5cm}|p{1.5cm}|p{1.5cm}|p{1.5cm}|}
    \hline
    \multicolumn{5}{|c|}{2. Povprečna sprejeta moč}\\
    \hline
    Index & $\frac{\omega}{\omega_0}$ & $\frac{B}{B_0}$ & $\delta$ izr. & $\frac{\overline{P}}{M_0 B_0 \omega_0}$\\
    \hline
    1 & 0,25 & 1,07 & 0,05 & 0,01\\
    2 & 0,50 & 1,33 & 0,13 & 0,04\\
    3 & 0,75 & 2,19 & 0,34 & 0,27\\
    4 & 0,80 & 2,58 & 0,43 & 0,43\\
    5 & 0,86 & 3,13 & 0,56 & 0,72\\
    6 & 0,91 & 3,90 & 0,78 & 1,25\\
    7 & 0,96 & 4,76 & 1,14 & 2,06\\
    8 & 0,98 & 5,00 & 1,37 & 2,40\\
    9 & 1,01 & 4,97 & 1,62 & 2,49 \\
    10 & 1,03 & 4,65 & 1,86 & 2,29\\
    11 & 1,06 & 4,17 & 2,07 & 1,93\\
    12 & 1,11 & 3,18 & 2,36 & 1,24\\
    13 & 1,16 & 2,45 & 2,54 & 0,80\\
    14 & 1,26 & 1,58 & 2,73 & 0,40\\
    15 & 1,51 & 0,76 & 2,91 & 0,13 \\
    \hline
\end{tabular}
\large\centering
\begin{tikzpicture}
\begin{axis}[
    title={Povprečna sprejeta moč brez dodatnega dušenja},
    xlabel={$\omega/\omega_0$},
    ylabel={$\frac{\overline{P}}{M_0 B_0 \omega_0}$},
    xmin=0, xmax=1.6,
    ymin=0, ymax=20,
    xtick={0,0.2,0.4,0.6,0.8,1,1.2,1.4,1.6},
    ytick={0,4,8,12,16,20},
    legend pos=north west,
    ymajorgrids=true,
    xmajorgrids=true,
    grid style=dashed,
]

\addplot[
    color=blue,
    mark=square,
    ]
    coordinates { (	0.251327412	,	0.008618424	)
(	0.502654825	,	0.053585133	)
(	0.753982237	,	0.337567979	)
(	0.804247719	,	0.545201959	)
(	0.879645943	,	1.320736235	)
(	0.904778684	,	1.900713369	)
(	0.929911425	,	2.900393859	)
(	0.942477796	,	3.70608328	)
(	0.955044167	,	4.908793818	)
(	1.005309649	,	19.04136956	)
(	1.055575132	,	4.111744655	)
(	1.105840614	,	1.886393404	)
(	1.256637061	,	0.500431672	)
(	1.507964474	,	0.16166852	)

    
    };
    \legend{eksperimentalni}

\addplot[
    color=red
    ]
    coordinates { 
(	0.00	,	0.00	)
(	0.02	,	0.00	)
(	0.03	,	0.00	)
(	0.05	,	0.00	)
(	0.06	,	0.00	)
(	0.08	,	0.00	)
(	0.10	,	0.00	)
(	0.11	,	0.00	)
(	0.13	,	0.00	)
(	0.14	,	0.00	)
(	0.16	,	0.00	)
(	0.18	,	0.00	)
(	0.19	,	0.00	)
(	0.21	,	0.01	)
(	0.22	,	0.01	)
(	0.24	,	0.01	)
(	0.26	,	0.01	)
(	0.27	,	0.01	)
(	0.29	,	0.01	)
(	0.30	,	0.01	)
(	0.32	,	0.02	)
(	0.34	,	0.02	)
(	0.35	,	0.02	)
(	0.37	,	0.02	)
(	0.38	,	0.02	)
(	0.40	,	0.03	)
(	0.42	,	0.03	)
(	0.43	,	0.03	)
(	0.45	,	0.04	)
(	0.46	,	0.04	)
(	0.48	,	0.05	)
(	0.50	,	0.05	)
(	0.51	,	0.06	)
(	0.53	,	0.06	)
(	0.54	,	0.07	)
(	0.56	,	0.08	)
(	0.58	,	0.09	)
(	0.59	,	0.10	)
(	0.61	,	0.11	)
(	0.62	,	0.12	)
(	0.64	,	0.14	)
(	0.66	,	0.15	)
(	0.67	,	0.17	)
(	0.69	,	0.20	)
(	0.70	,	0.22	)
(	0.72	,	0.25	)
(	0.74	,	0.29	)
(	0.75	,	0.33	)
(	0.77	,	0.38	)
(	0.78	,	0.45	)
(	0.80	,	0.52	)
(	0.82	,	0.62	)
(	0.83	,	0.73	)
(	0.85	,	0.88	)
(	0.86	,	1.08	)
(	0.88	,	1.33	)
(	0.90	,	1.66	)
(	0.91	,	2.13	)
(	0.93	,	2.80	)
(	0.94	,	3.83	)
(	0.96	,	5.56	)
(	0.98	,	9.04	)
(	0.99	,	17.27	)
(	1.01	,	17.32	)
(	1.02	,	9.23	)
(	1.04	,	5.79	)
(	1.06	,	4.08	)
(	1.07	,	3.06	)
(	1.09	,	2.40	)
(	1.10	,	1.93	)
(	1.12	,	1.59	)
(	1.14	,	1.33	)
(	1.15	,	1.13	)
(	1.17	,	0.98	)
(	1.18	,	0.85	)
(	1.20	,	0.75	)
(	1.22	,	0.66	)
(	1.23	,	0.59	)
(	1.25	,	0.53	)
(	1.26	,	0.48	)
(	1.28	,	0.43	)
(	1.30	,	0.40	)
(	1.31	,	0.36	)
(	1.33	,	0.33	)
(	1.34	,	0.31	)
(	1.36	,	0.29	)
(	1.38	,	0.27	)
(	1.39	,	0.25	)
(	1.41	,	0.23	)
(	1.42	,	0.22	)
(	1.44	,	0.21	)
(	1.46	,	0.19	)
(	1.47	,	0.18	)
(	1.49	,	0.17	)
(	1.50	,	0.16	)
(	1.52	,	0.16	)
(	1.54	,	0.15	)
(	1.55	,	0.14	)
(	1.57	,	0.13	)
(	1.58	,	0.13	)
(	1.60	,	0.12	)

    
    };
    
\addlegendentry{idealni}
    
\end{axis}
\end{tikzpicture}

\begin{tikzpicture}
\begin{axis}[
    title={Povprečna sprejeta moč z dodatnim dušenjem},
    xlabel={$\omega/\omega_0$},
    ylabel={$B/B_0$},
    xmin=0, xmax=1.6,
    ymin=0, ymax=20,
    xtick={0,0.2,0.4,0.6,0.8,1,1.2,1.4,1.6},
    ytick={0,4,8,12,16,20},
    legend pos=north west,
    ymajorgrids=true,
    xmajorgrids=true,
    grid style=dashed,
]

\addplot[
    color=blue,
    mark=square,
    ]
    coordinates { (	0.251327412	,	0.007176381	)
(	0.502654825	,	0.044434163	)
(	0.753982237	,	0.27208105	)
(	0.804247719	,	0.429455562	)
(	0.854513202	,	0.715849822	)
(	0.904778684	,	1.247103513	)
(	0.955044167	,	2.063169249	)
(	0.980176908	,	2.403628376	)
(	1.005309649	,	2.493008689	)
(	1.03044239	,	2.29367001	)
(	1.055575132	,	1.933768627	)
(	1.105840614	,	1.240342815	)
(	1.156106097	,	0.801546913	)
(	1.256637061	,	0.396205772	)
(	1.507964474	,	0.132675451	)

    };
    \legend{eksperimentalni}

\addplot[
    color=red
    ]
    coordinates { (	0.00	,	0.00	)
(	0.02	,	0.00	)
(	0.03	,	0.00	)
(	0.05	,	0.00	)
(	0.06	,	0.00	)
(	0.08	,	0.00	)
(	0.10	,	0.00	)
(	0.11	,	0.00	)
(	0.13	,	0.00	)
(	0.14	,	0.00	)
(	0.16	,	0.00	)
(	0.18	,	0.00	)
(	0.19	,	0.00	)
(	0.21	,	0.01	)
(	0.22	,	0.01	)
(	0.24	,	0.01	)
(	0.26	,	0.01	)
(	0.27	,	0.01	)
(	0.29	,	0.01	)
(	0.30	,	0.01	)
(	0.32	,	0.02	)
(	0.34	,	0.02	)
(	0.35	,	0.02	)
(	0.37	,	0.02	)
(	0.38	,	0.02	)
(	0.40	,	0.03	)
(	0.42	,	0.03	)
(	0.43	,	0.03	)
(	0.45	,	0.04	)
(	0.46	,	0.04	)
(	0.48	,	0.05	)
(	0.50	,	0.05	)
(	0.51	,	0.06	)
(	0.53	,	0.06	)
(	0.54	,	0.07	)
(	0.56	,	0.08	)
(	0.58	,	0.09	)
(	0.59	,	0.10	)
(	0.61	,	0.11	)
(	0.62	,	0.12	)
(	0.64	,	0.13	)
(	0.66	,	0.15	)
(	0.67	,	0.17	)
(	0.69	,	0.19	)
(	0.70	,	0.21	)
(	0.72	,	0.24	)
(	0.74	,	0.27	)
(	0.75	,	0.31	)
(	0.77	,	0.36	)
(	0.78	,	0.41	)
(	0.80	,	0.48	)
(	0.82	,	0.55	)
(	0.83	,	0.65	)
(	0.85	,	0.76	)
(	0.86	,	0.89	)
(	0.88	,	1.05	)
(	0.90	,	1.24	)
(	0.91	,	1.46	)
(	0.93	,	1.70	)
(	0.94	,	1.95	)
(	0.96	,	2.19	)
(	0.98	,	2.38	)
(	0.99	,	2.49	)
(	1.01	,	2.49	)
(	1.02	,	2.39	)
(	1.04	,	2.21	)
(	1.06	,	2.00	)
(	1.07	,	1.78	)
(	1.09	,	1.56	)
(	1.10	,	1.37	)
(	1.12	,	1.20	)
(	1.14	,	1.05	)
(	1.15	,	0.93	)
(	1.17	,	0.82	)
(	1.18	,	0.73	)
(	1.20	,	0.66	)
(	1.22	,	0.59	)
(	1.23	,	0.53	)
(	1.25	,	0.48	)
(	1.26	,	0.44	)
(	1.28	,	0.40	)
(	1.30	,	0.37	)
(	1.31	,	0.34	)
(	1.33	,	0.32	)
(	1.34	,	0.29	)
(	1.36	,	0.27	)
(	1.38	,	0.26	)
(	1.39	,	0.24	)
(	1.41	,	0.22	)
(	1.42	,	0.21	)
(	1.44	,	0.20	)
(	1.46	,	0.19	)
(	1.47	,	0.18	)
(	1.49	,	0.17	)
(	1.50	,	0.16	)
(	1.52	,	0.15	)
(	1.54	,	0.14	)
(	1.55	,	0.14	)
(	1.57	,	0.13	)
(	1.58	,	0.13	)
(	1.60	,	0.12	)
    
    };
    
\addlegendentry{idealni}
    
\end{axis}
\end{tikzpicture}

\raggedright
\normalsize

\pagebreak

%Zaključek
\section{Analiza rezultatov}\label{sec:zakljucek}

Ob pregledu rezultatov in napak, ugotovimo, da na koncu imamo napako. To je lahko iz raznih razlogov, naprimer da nihanje motorja, ki je poganjalo vzmet ni bilo zares sinusno, da ima motor sam napako, med zapisano frekvenco in dejansko frekvenco in podobno. Največja napaka je pri maksimalni meritvi zato, ker je odklon bil tako velik, da več ni bil na merilni lestvici, ter smo ga le aproksimirali na 180 deg.

%%%%%%%%%%%%%%%%%%%%%%%%%%%%%%%%%%%%%%%%%%%%%%%%%%%%%%%%%%
%\makebox[0pt][l]{\begin{minipage}{\textwidth}\centering\includegraphics[width=0.8\textwidth]{slike/232.png}\captionof{figure}{Graf meritev masa 1}\end{minipage}}

%\makebox[0pt][l]{\begin{minipage}{\textwidth}\centering\includegraphics[width=0.8\textwidth]{slike/680.png}\captionof{figure}{Graf meritev masa 2}\end{minipage}}

%\makebox[0pt][l]{\begin{minipage}{\textwidth}\centering\includegraphics[width=0.8\textwidth]{slike/762.png}\captionof{figure}{Graf meritev masa 3}\end{minipage}}
%%%%%%%%%%%%%%%%%%%%%%%%%%%%%%%%%%%%%%%%%%%%%%%%%%%%%%%%%%
\pagebreak

\end{document} 