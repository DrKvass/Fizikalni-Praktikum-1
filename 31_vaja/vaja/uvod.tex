\section{Teoretični uvod}\label{sec:uvod}
\subsection{Lastno nihanje torzijskega nihala}
Nihanje je periodično gibanje okoli stabilne ravnovesne lege. nedušeno nihanje opišemo z enačbo:

\centering \Large
\begin{equation}
    \Phi = A\sin(\omega_0t); \omega_0 = \sqrt{D/J} 
\end{equation}
\raggedright \normalsize

V resnici ne obstaja nedušeno nihanje, saj se zaradi različnih uporov energija nihala manjša, to dušeno nihanje opišemo z enačbo:


\centering \Large
\begin{equation}
    \Phi = A_0e^{-\beta t}\sin(\omega_d t); \omega_d = \sqrt{\omega_0^2-\beta^2}
\end{equation}
\raggedright \normalsize

 iz tega sledi:
 
\centering \Large
\begin{equation}
    A_n = A_0e^{-\frac{2\pi\beta n}{\omega_d}}
\end{equation}
\raggedright \normalsize

z logaritmi dobimo:

\centering \Large
\begin{equation}
    \beta = \frac{\omega_d}{2\pi n}\ln{\frac{A_0}{A_n}}
\end{equation}
\raggedright \normalsize

\subsection{Vsiljeno nihanje torzijskega nihala}

Pri lastnem nihanju torzijskega nihala, po začetnim sunkom navora, frekvenco določajo lastnosti nihala ($J, D, \beta$), amplituda pa je odvisna od začetnega nihanja.
Na nihalo pa lahko deluje tudi sinusno nihajoč zunanji izvor navora, za katerega velja:

\centering \Large
\begin{equation}
    M = M_0*\sin(\omega t)
\end{equation}
\raggedright \normalsize

, se kmalu ustali s frekvenco tega navora, sicer ne v isti fazi. Takrat pravimo, da nihalo vsiljeno niha, kar opišemo z izrazom:

\centering \Large
\begin{equation}
    \Phi = B \sin(\omega t - \delta)
\end{equation}
\raggedright \normalsize

Amplituda $B$ je odvisna od amplitude $M_0$ navora M, razen tega pa od krožne frekvence $\beta$ in od lastnosti nihala. Za torzijsko nihalo z vsiljenim nihanjem velja tudi:

\centering \Large
\begin{equation}
    B = \frac{B_0}{\sqrt{[1-(\frac{\omega}{\omega_0})^2]^2 + a^2(\frac{\omega}{\omega_0})^2}}
\end{equation}
\begin{equation}
    B_0 = \frac{M_0}{D}
\end{equation}
\begin{equation}
    a = \frac{2\beta}{\omega_0}
\end{equation}
\begin{equation}
     \tan(\delta) = \frac{a(\frac{\omega}{\omega_0})}{1-(\frac{\omega}{\omega_0})^2}
\end{equation}
\raggedright \normalsize

Amplituda B je konstanta, ker dovajana moč P, s povprečno vrednostjo

\centering \Large
\begin{equation}
    \overline{P} = \frac{\omega}{2\pi} \int_0^{2\pi} Md\phi= \frac{1}{2}\omega M_0 B \sin(\delta)
\end{equation}
\raggedright \normalsize

ravno pokriva energijske izgube nihala.
Če spreminjamo frekvenco zunanjega navora $M$ , lahko zasledujemo odvisnost amplitude $B$ od frekvence. Pri majhnih ($\frac{\omega}{\omega_0} \ll 1$) je odmik nihala iz ravnovesne lege ves čas skoraj sorazmeren z navorom $M$ , amplituda pa je $B = B_0 = \frac{M_0}{D}$; fazna razlika med obema nihanjima je zelo majhna ($\delta \approx 0$). Ko povečujemo frekvenco navora, se amplituda povečuje in doseže pri $\omega \approx \omega_0$ največjo vrednost. Pravimo, de je tedaj nihalo v resonanci z navorom. Fazni premik je tedaj približno $90$ $deg$ nihalo zaostaja za navorom za četrt nihaja. Poprečna moč (glej enačbo), ki jo nihalo sprejema, je v okolici resonance največja. Pri nadaljnjem večanju frekvence začne amplituda padati, nihalo pa vedno bolj zaostaja za
navorom. Pri $\omega \ll \omega_0$ gre amplituda $B$ proti nič, fazni premik pa proti $180$ $deg$.
Krivuljo, ki nam kaže odvisnost razmerja $B/B_0$ od frekvence navora, imenujemo
resonančna krivulja 

